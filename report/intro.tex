\section{Εισαγωγή}

Στην παρούσα εργασία επεξεργαζόμαστε ιστορικά δεδομένα από το χρηματιστήριο της Νέας Υόρκης για τα έτη $1962$ -- $2017$.

Το σύνολο δεδομένων που μας διατίθεται στην αρχική του μορφή αποτελείται από ένα σύνολο αρχείων CSV, καθένα από τα οποία αναφέρεται σε μία εταιρεία του χρηματιστηρίου.
Οι στήλες κάθε τέτοιου αρχείου είναι:
\begin{itemize}
    \item η ημερομηνία,
    \item η τιμή ανοίγματος της μετοχής την δεδομένη ημερομηνία,
    \item η τιμή κλεισίματος της μετοχής την δεδομένη ημερομηνία,
    \item η ελάχιστη τιμή της μετοχής που καταγράφθηκε την δεδομένη ημερομηνία,
    \item η μέγιστη τιμή της μετοχής που καταγράφθηκε την δεδομένη ημερομηνία, και
    \item ο συνολικός όγκος των μετοχών που διαπραγματεύτηκαν την δεδομένη ημερομηνία.
\end{itemize}

Στόχος της εργασίας είναι να καταφέρουμε να παράγουμε δύο ακολουθίες αγοραπωλησιών μετοχών εταιρειών, μία μικρή και μία μεγάλη, με τους δεδομένους περιορισμούς που περιγράφονται στην εκφώνηση της άσκησης.

Λόγω της εγγενούς δυσκολίας και της αυξημένης πολυπλοκότητας (NP-hard) του προβλήματος, καθώς και του ικανού μεγέθους του συνόλου δεδομένων εισόδου, δεν επιχειρήθηκε η εύρεση βέλτιστης λύσης, παρά μόνο μίας ικανοποιητικής (ανά υποπρόβλημα, δηλαδή ανά μέγεθος ακολουθίας) λύσης σύμφωνα με τον παρεχόμενο online επικυρωτή.

Στις επόμενες ενότητες εξηγούμε εν συντομία τη συλλογιστική του κάθε αλγορίθμου που σχεδιάστηκε και υλοποιήθηκε, αφού εξηγήσουμε τον τρόπο εκτέλεσης του παραδοτέου script για την αναπαραγωγή των αποτελεσμάτων.

Ο κώδικας της υλοποίησης, των σχημάτων, καθώς και του παρόντος εγγράφου (για το οποίο χρησιμοποιήθηκε περιβάλλον \LaTeX), είναι δημοσιοποιημένα σε αποθετήριο του GitHub και προσπελάσιμο μέσω του υπερσυνδέσμου: \url{https://github.com/ckatsak/progds2020-timetravel}.
